% Article template for Mathematics Magazine
% Revised 7/2002  Thanks for Greg St. George
\documentclass[12pt]{article}
\usepackage{amssymb}
\usepackage[ngerman]{babel}
\usepackage[utf8]{inputenc}
\usepackage{amsmath}
\usepackage{pf2}
\usepackage{graphicx}
\renewcommand{\baselinestretch}{1.2}
%This is the command that spaces the manuscript for easy reading
\newtheorem{zeige}{Zeige}



%todo 
\usepackage[colorinlistoftodos,prependcaption,textsize=tiny]{todonotes}
\usepackage{xargs}                      % Use more than one optional parameter in a new commands
\newcommandx{\QUESTION}[2][1=]{\todo[linecolor=none,backgroundcolor=blue!15,bordercolor=none,#1]{\textbf{QUESTION: }#2}}



\begin{document}
%\thispagestyle{empty}
\begin{center}
\Large
% TITLE GOES HERE
Logik und Komplexität  \textsc{ Übung 7 }
\end{center}

\begin{flushright}
Denis Erfurt, 532437\\
HU Berlin \\

\vspace{2 mm}

\end{flushright}

\subsubsection*{Aufgabe 1)}

Aus Übungsblatt 1 Aufgabe 1 wissen wir, dass es für eine beliebige Relationalle
Signatur $\sigma$ und einer $\sigma$-Struktur $\mathfrak{A}$ eine Isomorphe 
$\hat \sigma$-Struktur $\mathfrak{A}'$ wobei $\hat \sigma$ eine binäre Signatur
ist. Sei O.b.d.A. $\hat \sigma = \{ R_1, ..., R_m\}$

Um zu Zeigen: für alle $l\in\mathbb{N}$ und für alle $F\subseteq 
\Delta_{l+1}^\sigma$ gilt:
\[ \mu(EA_{l,F}|All(\sigma)) = 1 \] 
genügt es zu zeigen:
\[ \mu(\neg EA_{l,F}|All(\hat \sigma)) = 0 \]

Zunächst schätzen wir $|C_n|$ ab:
\begin{enumerate}
  \item Es gibt ${n \choose l}$ Möglichkeiten, die Menge $T$ zu Wählen.
  \item Es gibt ${n-l \choose l}$ Möglichkeiten, die Menge $T'$ zu Wählen.
  \item Es gibt $m * 2^{2l \choose 2}$ Möglichkeiten Relationen über Knoten aus
    $T\cup T'$ zu wählen.
  \item Es gibt $m * 2^{n-2l \choose 2}$ Möglichkeiten Relationen zwischen 
    Knoten aus $V\setminus T \cup T'$ zu wählen.
  \item Für jeden der $(n-2l)$ Knoten z in $V\setminus (T\cup T')$ gibt es 
    $(m*2^{2l})^{n-2l}-1$ Möglichkeiten Relationen mit Knoten aus $(T\cup T')$ 
    zu bilden, ohne den einen Fall, der das Erweiterungsaqiom erfüllt.
\end{enumerate}

\[ |C_n| \leq {n\choose l} * {n-l \choose l} * 2^{2l \choose 2} * 
  2^{n-2l\choose 2} * ( m * 2^{2l} - 1 )^{ n - 2l } * 2m  \] 

Durch Umformung erhalten wir:

\[ \frac{|C_n|}{|All(\hat \sigma)|} \leq n^{2l}*2m*(\frac{m*2^{2l}-1}{2^{2l}})^{n-2l} \] 

\[ n^{2l}*2m*(\frac{m*2^{2l}-1}{2^{2l}})^{n-2l} \longrightarrow_{n \rightarrow \infty} 0\] 

Damit ist gezeigt: 

\[ \mu(\neg EA_{l,F}|All(\hat \sigma)) = 0 \]
Sowie
\[ \mu(EA_{l,F}|All(\sigma)) = 1 \]
\qed

\subsubsection*{Aufgabe 2)}


\subsubsection*{Aufgabe 3)}

\subsubsection*{Aufgabe 4)}






\end{document}
