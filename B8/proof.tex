% Article template for Mathematics Magazine
% Revised 7/2002  Thanks for Greg St. George
\documentclass[12pt]{article}
\usepackage{amssymb}
\usepackage[ngerman]{babel}
\usepackage[utf8]{inputenc}
\usepackage{amsmath}
\usepackage{amsthm}
\renewcommand{\baselinestretch}{1.2}
%This is the command that spaces the manuscript for easy reading
\newtheorem{zeige}{Zeige}


%todo 
\usepackage[colorinlistoftodos,prependcaption,textsize=tiny]{todonotes}
\usepackage{xargs}                      % Use more than one optional parameter in a new commands
\newcommandx{\QUESTION}[2][1=]{\todo[linecolor=none,backgroundcolor=blue!15,bordercolor=none,#1]{\textbf{QUESTION: }#2}}



\begin{document}
%\thispagestyle{empty}
\begin{center}
\Large
% TITLE GOES HERE
Logik und Komplexität  \textsc{ Übung 7 }
\end{center}

\begin{flushright}
Denis Erfurt, 532437\\
HU Berlin \\

\vspace{2 mm}

\end{flushright}

\subsubsection*{Aufgabe 1)}
\subsubsection*{Aufgabe 2)}
Sei $GG\subseteq UGraph$ die Klasse der Strukturen, bei denen jeder Knoten 
einen Geraden Grad besitzt. 

zeige $GG$ ist nicht EMSO-definierbar in $UGraph$.

Laut dem Satz von Ajtai und Fagin genügt es zu zeigen, dass Duplicator eine 
Gewinnstrategie im (l,m)-Ajtai-Fagin-Spiel besitzt.

\textbf{Phase 1.} Duplicator wählt einen vollständigen-Graphen $\mathfrak{A} = K_{2^{l+m}+1}$.

\textbf{Beobachtung:} für einen vollständigen-Graphen gilt:
  \[ \text{n ist ungerade} \Leftrightarrow K_n \in GG \] 
  
Spoiler wählt hiernach die Mengen $X_1^\mathfrak{A}, ..., X_l^\mathfrak{A} 
\subseteq V$

Sei $c^\mathfrak{A}(a) := \{ X_i^\mathfrak{A} : a\in X_i^\mathfrak{A} \}$ die
Farbe eines Knotens $a$. 

Für jede Farbe $f\subseteq \{ X_1^\mathfrak{A}, ..., X_l^\mathfrak{A} \}$ sei

\[ M_f^\mathfrak{A} := \{ a \in A : c^\mathfrak{A}(a) = f \} \] 

\textbf{zeige: } nach l Mengen exestiert exestiert ein $M_f^\mathfrak{A}$ so 
dass $|M_f^\mathfrak{A}| \geq 2^m$

  Beweis durch vollständige Induktion: 
  
  \textbf{Induktionsannahme:} nach der i-ten Menge $X_i^\mathfrak{A}$ exestiert
  ein $f$ mit $|M_f^\mathfrak{A}| \geq 2^{l-i+m}$

  \textbf{Induktionsanfang:} $i=0$ Wir wissen dass $|A| = 2^{l+m}$. Für $f=\{\}$
  ist $M_f^\mathfrak{A} = A \Rightarrow |M_f^\mathfrak{A}|\geq 2^{l+m}$

  \textbf{Induktionsschritt:} $i \rightarrow i+1$
  Nach \textbf{IA} exestiert ein $f$ mit $|M_f^\mathfrak{A}|\geq 2^{l-i+m}$
  Spoiler wählt ein $X_{i+1}^\mathfrak{A}$.

  Sei $f' := f \cup \{X_{i+1}^\mathfrak{A}\}$

  Nach \textbf{IA} wissen wir:
  \[ |M_{f'}^\mathfrak{A}| + |M_f^\mathfrak{A}| \geq 2^{l-i+m} \] 

  Falls $|M_{f'}^\mathfrak{A}| < 2^{l-(i+1)+m}$, dann folgt daraus 
  $|M_f^\mathfrak{A}|\geq 2^{m-(i+1)+m}$

  Falls $|M_f^\mathfrak{A}| < 2^{l-(i+1)+m}$, dann folgt daraus 
  $|M_{f'}^\mathfrak{A}|\geq 2^{m-(i+1)+m}$

  \textbf{Indunktionsschluss:} Nach l Mengen exestiert eine Farbe f mit 
  $|M_f^\mathfrak{A}|\geq 2^{m}$
  
\textbf{Phase 2.} Duplicator wählt $\mathfrak{B} = K_{2^{l+m}+2}$. Nach Beobachtung
ist $\mathfrak{B}\in UGraph \setminus GG$
Weiter wählt Duplicator die Mengen $X_1^\mathfrak{B}, ..., X_l^\mathfrak{B}$ so, 
dass für jede Farbe $f$ gilt:
\begin{equation}
  |M_f^\mathfrak{B}|=|M_f^\mathfrak{A}|\text{ oder } |M_f^\mathfrak{B}|,|M_f^\mathfrak{A}|
  \geq 2^m \label{eq}
\end{equation}
Intuitiv färbt Duplicator den neuen Knoten mit der in $\mathfrak{A}$ häufigsten Farbe.

\textbf{Phase 3.} Betrachte das EF-Spiel auf $\mathfrak{A}' := (\mathfrak{A}, 
X_1^\mathfrak{A}, ... , X_l^\mathfrak{A})$ und $\mathfrak{B}' := (\mathfrak{B}, 
X_1^\mathfrak{B}, ... , X_l^\mathfrak{B})$

Für jede Wahl $a_i\in A$ von Spoiler kann Dup wegen (\ref{eq}) ein $b_i\in B$ wählen,
so dass $c(a)^\mathfrak{A} = c(b)^\mathfrak{B}$:
Falls $|M_f^\mathfrak{B}|=|M_f^\mathfrak{A}|$ so hat Duplicator eine Gewinnstrategie, 
in dem er Spoilers züge Kopiert. Falls $|M_f^\mathfrak{B}|,|M_f^\mathfrak{A}|
\geq 2^m $ so besitzt Duplicator eine Gewinnstrategie, indem er ein neues Element wählt,
falls Spoiler ein neues Element mit dieser Farbe gewählt hat. Andernfalls falls Spoiler
ein in Runde i gewähltes Element wählt, so wählt Duplicator in Runde i gewählte Element
der anderen Struktur. 
Analog für Spoilers wahl aus $\mathfrak{B}$.

Somit ist gezeigt das Duplicator eine Gewinnstrategie im (l,m)-Ajtai-Fagin-Spiel
besitzt. Somit ist nach Satz 3.44 $GG$ nicht EMSO-definierbar in $UGraph$. \qed






\subsubsection*{Aufgabe 3)}
\subsubsection*{Aufgabe 4)}

\end{document}
