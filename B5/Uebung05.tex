% Article template for Mathematics Magazine
% Revised 7/2002  Thanks for Greg St. George
\documentclass[12pt]{article}
\usepackage{amssymb}
\usepackage[ngerman]{babel}
\usepackage[utf8]{inputenc}
\usepackage{amsmath}
\usepackage{amsthm}
\renewcommand{\baselinestretch}{1.2}
%This is the command that spaces the manuscript for easy reading
\newtheorem{zeige}{Zeige}

\begin{document}
%\thispagestyle{empty}
\begin{center}
\Large
% TITLE GOES HERE
Logik und Komplexität  \textsc{ Übung 5 }
\end{center}

\begin{flushright}
Notizen zur fünften Übung\\
HU Berlin \\

\vspace{2 mm}

\end{flushright}

\subsubsection*{Aufgabe 1)}
\subsubsection*{Aufgabe 2)}
\begin{zeige}
  Falls $|A| < |B|$ und $|A| \leq 2^m$ so hat Spoiler eine Gewinnstrategie in m-Runden EF-Spiel auf $\mathcal{A}$ und $\mathcal{B}$.
\end{zeige}

\paragraph{$(**)_i$}: Sind $a_{2+1},...,a_{2+i}$ und $b_{2+1},..,b_{2+i}$ die in Runden $1, ..., i$ gewählten Elemente in A und B, so gibt es $j,j'\in {1,...,2+i}$ so dass gilt: \\
  1. $(a_j <^A a_{j'}\text{ und }b_j \geq^B b_j )\text{ oder }(a_j \geq^A a_{j'}\text{ und } b_j <^B b_{j'}) $\\
  oder\\
  2. $Dist(a_j,a_{j'})<2^{m-i}\text{ und } Dist(a_j,a_{j'})<Dist(b_j,b_{j'})$

\begin{zeige}
  $(**)_i$ gilt für jeden Schritt i: \\

  \textbf{IV:} $|A| < |B|$ und $|A| \leq 2^m$ \\
  \\
  \textbf{IA: $i=0$} $(**)_0$ gilt, durch die IV \\
  \textbf{IS: $i\rightarrow i+1:$} \\
  Sp. wählt in Runde $i+1$ das Element $b_{i+1}$ mit $b_j<b_{i+1}<b_{j'}$ mit $Dist(b_j,b_{i+1})=2^{m-(i+1)}$ \\
  Seien $a_j,a_{j'}$ jeweils die in Runde $j,j'$ von Dup gespielten Elemente. \\
  \textbf{zeige: } f.a. $a_{i+1}$ gilt $(**)_{i+1}$ \\
  \textbf{ Fall 1: } $a_{i+1} = a_j$ \\
    $Dist(a_{i+1},a_j) = 0 < 2^{m-(i+1)}$ \\
    $\Rightarrow (**)_{i+1}$ gilt \\
  \textbf{ Fall 2: } $a_{i+1} < a_j$ \\
  gilt nach $(**)_{i+1}.1$ \\
  \textbf{ Fall 3: } $a_{i+1} > a_{j'}$ \\ 
  gilt nach $(**)_{i+1}.1$ \\
  \textbf{ Fall 4: } $a_{j} < a_{i+1} < a_{j'}$ \\
  \textbf{ Fall 4.1: } $dist(a_j,a_{i+1}) = dist(b_j,b_{i+1})$\\
  Da $da_{j,j'} < db_{j,j'} \Rightarrow da_{i+1,j'}<db_{i+1,j'}$\\
  Da $db_{i+1,j'}\geq 2^{m-(i+1)} \Rightarrow da_{i+1,j'}<2^{m-(i+1)}$ \\
  $\Rightarrow (**)_{i+1}$ gilt \\
  \textbf{ Fall 4.2: } $dist(a_j,a_{i+1}) < dist(b_j,b_{i+1})$\\
  Da $db_{i+1,j'} \geq 2^{m-(i+1)} \Rightarrow da_{j,i+1}<2^{m-(i+1)}$\\
  $\Rightarrow (**)_{i+1}$ gilt \\
  \textbf{ Fall 4.3: } $dist(a_j,a_{i+1}) > dist(b_j,b_{i+1})$ \\
  $da_{i+1,j} < 2^{m-i} - da_{j,i+1}$ und $da_{j,i+1} > 2^{m-(i+1)}$ \\
  $\Rightarrow 2^{m-i} - da_{j,i+1} < 2^{m-(i+1)}$
  $\Rightarrow (**)_{i+1}$ gilt \\

  
  
  
  

\end{zeige}
\subsubsection*{Aufgabe 3)}

\textbf{Definition Aquivalenz für Formeln: }\\
$ \phi \approx_m \psi \Leftrightarrow \text{ für jede Äquivalenzklasse } [(A,\bar a)]_{\approx_m} \text{ so dass für alle } A\in [(A,\bar a)]_{\approx_m}\text{ gilt:}$
\[ A\models \phi \Leftrightarrow A\models \psi \] \\
Da es für jede Klasse eine Hintikkaformel gibt die Sie beschreibt gilt für alle $\phi\in FO[\sigma] $ mit $qr(\phi) = m\text{ ein } Q: \phi \equiv \bigvee_{h\in Q\subseteq m-Typen_k[\sigma]} h$. \\
Da $m-Typen_k[\sigma]$ endlich ist und die Potenzmenge einer endlichen Menge ebenfalls endlich ist, gibt es endlich viele Q's.\\
Die Vereinigung ($\cup$) über k sowie über m von $m-Typen_k[\sigma]$ liefert nur ebenfalls nur endlich viele Elemente. Somit ist auch bis auf die logische Äquivalenz die menge der $FO[\sigma]$ mit $qr\leq m$ und $frei\subseteq {x_1,...,x_k}$ endlich.


\subsubsection*{Aufgabe 4)}

\end{document}
